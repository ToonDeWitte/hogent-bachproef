%===============================================================================
% LaTeX sjabloon voor de bachelorproef toegepaste informatica aan HOGENT
% Meer info op https://github.com/HoGentTIN/latex-hogent-report
%===============================================================================

\documentclass[dutch,dit,thesis]{hogentreport}

% TODO:
% - If necessary, replace the option `dit`' with your own department!
%   Valid entries are dbo, dbt, dgz, dit, dlo, dog, dsa, soa
% - If you write your thesis in English (remark: only possible after getting
%   explicit approval!), remove the option "dutch," or replace with "english".

\usepackage{lipsum} % For blind text, can be removed after adding actual content

%% Pictures to include in the text can be put in the graphics/ folder
\graphicspath{{graphics/}}

%% For source code highlighting, requires pygments to be installed
%% Compile with the -shell-escape flag!
\usepackage[section]{minted}
\usemintedstyle{solarized-light}
\definecolor{bg}{RGB}{253,246,227} %% Set the background color of the codeframe

%% Change this line to edit the line numbering style:
\renewcommand{\theFancyVerbLine}{\ttfamily\scriptsize\arabic{FancyVerbLine}}

%% Macro definition to load external java source files with \javacode{filename}:
\newmintedfile[javacode]{java}{
    bgcolor=bg,
    fontfamily=tt,
    linenos=true,
    numberblanklines=true,
    numbersep=5pt,
    gobble=0,
    framesep=2mm,
    funcnamehighlighting=true,
    tabsize=4,
    obeytabs=false,
    breaklines=true,
    mathescape=false
    samepage=false,
    showspaces=false,
    showtabs =false,
    texcl=false,
}

% Other packages not already included can be imported here

%%---------- Document metadata -------------------------------------------------
% TODO: Replace this with your own information
\author{Ernst Aarden}
\supervisor{Dhr. F. Van Houte}
\cosupervisor{Mevr. S. Beeckman}
\title[Optionele ondertitel]%
    {Titel van de bachelorproef}
\academicyear{\advance\year by -1 \the\year--\advance\year by 1 \the\year}
\examperiod{1}
\degreesought{\IfLanguageName{dutch}{Professionele bachelor in de toegepaste informatica}{Bachelor of applied computer science}}
\partialthesis{false} %% To display 'in partial fulfilment'
%\institution{Internshipcompany BVBA.}

%% Add global exceptions to the hyphenation here
\hyphenation{back-slash}

%% The bibliography (style and settings are  found in hogentthesis.cls)
\addbibresource{bachproef.bib}            %% Bibliography file
\addbibresource{../voorstel/voorstel.bib} %% Bibliography research proposal
\defbibheading{bibempty}{}

%% Prevent empty pages for right-handed chapter starts in twoside mode
\renewcommand{\cleardoublepage}{\clearpage}

\renewcommand{\arraystretch}{1.2}

%% Content starts here.
\begin{document}

%---------- Front matter -------------------------------------------------------

\frontmatter

\hypersetup{pageanchor=false} %% Disable page numbering references
%% Render a Dutch outer title page if the main language is English
\IfLanguageName{english}{%
    %% If necessary, information can be changed here
    \degreesought{Professionele Bachelor toegepaste informatica}%
    \begin{otherlanguage}{dutch}%
       \maketitle%
    \end{otherlanguage}%
}{}

%% Generates title page content
\maketitle
\hypersetup{pageanchor=true}

%%=============================================================================
%% Voorwoord
%%=============================================================================

\chapter*{\IfLanguageName{dutch}{Woord vooraf}{Preface}}%
\label{ch:voorwoord}

%% TODO:
%% Het voorwoord is het enige deel van de bachelorproef waar je vanuit je
%% eigen standpunt (``ik-vorm'') mag schrijven. Je kan hier bv. motiveren
%% waarom jij het onderwerp wil bespreken.
%% Vergeet ook niet te bedanken wie je geholpen/gesteund/... heeft

\lipsum[1-3]
%%=============================================================================
%% Samenvatting
%%=============================================================================

% TODO: De "abstract" of samenvatting is een kernachtige (~ 1 blz. voor een
% thesis) synthese van het document.
%
% Een goede abstract biedt een kernachtig antwoord op volgende vragen:
%
% 1. Waarover gaat de bachelorproef?
% 2. Waarom heb je er over geschreven?
% 3. Hoe heb je het onderzoek uitgevoerd?
% 4. Wat waren de resultaten? Wat blijkt uit je onderzoek?
% 5. Wat betekenen je resultaten? Wat is de relevantie voor het werkveld?
%
% Daarom bestaat een abstract uit volgende componenten:
%
% - inleiding + kaderen thema
% - probleemstelling
% - (centrale) onderzoeksvraag
% - onderzoeksdoelstelling
% - methodologie
% - resultaten (beperk tot de belangrijkste, relevant voor de onderzoeksvraag)
% - conclusies, aanbevelingen, beperkingen
%
% LET OP! Een samenvatting is GEEN voorwoord!

%%---------- Nederlandse samenvatting -----------------------------------------
%
% TODO: Als je je bachelorproef in het Engels schrijft, moet je eerst een
% Nederlandse samenvatting invoegen. Haal daarvoor onderstaande code uit
% commentaar.
% Wie zijn bachelorproef in het Nederlands schrijft, kan dit negeren, de inhoud
% wordt niet in het document ingevoegd.

\IfLanguageName{english}{%
\selectlanguage{dutch}
\chapter*{Samenvatting}
\lipsum[1-4]
\selectlanguage{english}
}{}

%%---------- Samenvatting -----------------------------------------------------
% De samenvatting in de hoofdtaal van het document

\chapter*{\IfLanguageName{dutch}{Samenvatting}{Abstract}}

\lipsum[1-4]


%---------- Inhoud, lijst figuren, ... -----------------------------------------

\tableofcontents

% In a list of figures, the complete caption will be included. To prevent this,
% ALWAYS add a short description in the caption!
%
%  \caption[short description]{elaborate description}
%
% If you do, only the short description will be used in the list of figures

\listoffigures

% If you included tables and/or source code listings, uncomment the appropriate
% lines.
%\listoftables
%\listoflistings

% Als je een lijst van afkortingen of termen wil toevoegen, dan hoort die
% hier thuis. Gebruik bijvoorbeeld de ``glossaries'' package.
% https://www.overleaf.com/learn/latex/Glossaries

%---------- Kern ---------------------------------------------------------------

\mainmatter{}

% De eerste hoofdstukken van een bachelorproef zijn meestal een inleiding op
% het onderwerp, literatuurstudie en verantwoording methodologie.
% Aarzel niet om een meer beschrijvende titel aan deze hoofdstukken te geven of
% om bijvoorbeeld de inleiding en/of stand van zaken over meerdere hoofdstukken
% te verspreiden!

%%=============================================================================
%% Inleiding
%%=============================================================================

\chapter{\IfLanguageName{dutch}{Inleiding}{Introduction}}%
\label{ch:inleiding}

De inleiding moet de lezer net genoeg informatie verschaffen om het onderwerp te begrijpen en in te zien waarom de onderzoeksvraag de moeite waard is om te onderzoeken. In de inleiding ga je literatuurverwijzingen beperken, zodat de tekst vlot leesbaar blijft. Je kan de inleiding verder onderverdelen in secties als dit de tekst verduidelijkt. Zaken die aan bod kunnen komen in de inleiding~\autocite{Pollefliet2011}:

\begin{itemize}
  \item context, achtergrond
  \item afbakenen van het onderwerp
  \item verantwoording van het onderwerp, methodologie
  \item probleemstelling
  \item onderzoeksdoelstelling
  \item onderzoeksvraag
  \item \ldots
\end{itemize}

\section{\IfLanguageName{dutch}{Probleemstelling}{Problem Statement}}%
\label{sec:probleemstelling}

Uit je probleemstelling moet duidelijk zijn dat je onderzoek een meerwaarde heeft voor een concrete doelgroep. De doelgroep moet goed gedefinieerd en afgelijnd zijn. Doelgroepen als ``bedrijven,'' ``KMO's'', systeembeheerders, enz.~zijn nog te vaag. Als je een lijstje kan maken van de personen/organisaties die een meerwaarde zullen vinden in deze bachelorproef (dit is eigenlijk je steekproefkader), dan is dat een indicatie dat de doelgroep goed gedefinieerd is. Dit kan een enkel bedrijf zijn of zelfs één persoon (je co-promotor/opdrachtgever).

\section{\IfLanguageName{dutch}{Onderzoeksvraag}{Research question}}%
\label{sec:onderzoeksvraag}

Wees zo concreet mogelijk bij het formuleren van je onderzoeksvraag. Een onderzoeksvraag is trouwens iets waar nog niemand op dit moment een antwoord heeft (voor zover je kan nagaan). Het opzoeken van bestaande informatie (bv. ``welke tools bestaan er voor deze toepassing?'') is dus geen onderzoeksvraag. Je kan de onderzoeksvraag verder specifiëren in deelvragen. Bv.~als je onderzoek gaat over performantiemetingen, dan 

\section{\IfLanguageName{dutch}{Onderzoeksdoelstelling}{Research objective}}%
\label{sec:onderzoeksdoelstelling}

Wat is het beoogde resultaat van je bachelorproef? Wat zijn de criteria voor succes? Beschrijf die zo concreet mogelijk. Gaat het bv.\ om een proof-of-concept, een prototype, een verslag met aanbevelingen, een vergelijkende studie, enz.

\section{\IfLanguageName{dutch}{Opzet van deze bachelorproef}{Structure of this bachelor thesis}}%
\label{sec:opzet-bachelorproef}

% Het is gebruikelijk aan het einde van de inleiding een overzicht te
% geven van de opbouw van de rest van de tekst. Deze sectie bevat al een aanzet
% die je kan aanvullen/aanpassen in functie van je eigen tekst.

De rest van deze bachelorproef is als volgt opgebouwd:

In Hoofdstuk~\ref{ch:stand-van-zaken} wordt een overzicht gegeven van de stand van zaken binnen het onderzoeksdomein, op basis van een literatuurstudie.

In Hoofdstuk~\ref{ch:methodologie} wordt de methodologie toegelicht en worden de gebruikte onderzoekstechnieken besproken om een antwoord te kunnen formuleren op de onderzoeksvragen.

% TODO: Vul hier aan voor je eigen hoofstukken, één of twee zinnen per hoofdstuk

In Hoofdstuk~\ref{ch:conclusie}, tenslotte, wordt de conclusie gegeven en een antwoord geformuleerd op de onderzoeksvragen. Daarbij wordt ook een aanzet gegeven voor toekomstig onderzoek binnen dit domein.
\chapter{\IfLanguageName{dutch}{Stand van zaken}{State of the art}}%
\label{ch:stand-van-zaken}

% Tip: Begin elk hoofdstuk met een paragraaf inleiding die beschrijft hoe
% dit hoofdstuk past binnen het geheel van de bachelorproef. Geef in het
% bijzonder aan wat de link is met het vorige en volgende hoofdstuk.

% Pas na deze inleidende paragraaf komt de eerste sectiehoofding.

Dit hoofdstuk bevat je literatuurstudie. De inhoud gaat verder op de inleiding, maar zal het onderwerp van de bachelorproef *diepgaand* uitspitten. De bedoeling is dat de lezer na lezing van dit hoofdstuk helemaal op de hoogte is van de huidige stand van zaken (state-of-the-art) in het onderzoeksdomein. Iemand die niet vertrouwd is met het onderwerp, weet nu voldoende om de rest van het verhaal te kunnen volgen, zonder dat die er nog andere informatie moet over opzoeken \autocite{Pollefliet2011}.

Je verwijst bij elke bewering die je doet, vakterm die je introduceert, enz.\ naar je bronnen. In \LaTeX{} kan dat met het commando \texttt{$\backslash${textcite\{\}}} of \texttt{$\backslash${autocite\{\}}}. Als argument van het commando geef je de ``sleutel'' van een ``record'' in een bibliografische databank in het Bib\LaTeX{}-formaat (een tekstbestand). Als je expliciet naar de auteur verwijst in de zin, gebruik je \texttt{$\backslash${}textcite\{\}}.
Soms wil je de auteur niet expliciet vernoemen, dan gebruik je \texttt{$\backslash${}autocite\{\}}. In de volgende paragraaf een voorbeeld van elk.

\textcite{Knuth1998} schreef een van de standaardwerken over sorteer- en zoekalgoritmen. Experten zijn het erover eens dat cloud computing een interessante opportuniteit vormen, zowel voor gebruikers als voor dienstverleners op vlak van informatietechnologie~\autocite{Creeger2009}.

\lipsum[7-20]

%%=============================================================================
%% Methodologie
%%=============================================================================

\chapter{\IfLanguageName{dutch}{Methodologie}{Methodology}}%
\label{ch:methodologie}

%% TODO: Hoe ben je te werk gegaan? Verdeel je onderzoek in grote fasen, en
%% licht in elke fase toe welke stappen je gevolgd hebt. Verantwoord waarom je
%% op deze manier te werk gegaan bent. Je moet kunnen aantonen dat je de best
%% mogelijke manier toegepast hebt om een antwoord te vinden op de
%% onderzoeksvraag.

\lipsum[21-25]



% Voeg hier je eigen hoofdstukken toe die de ``corpus'' van je bachelorproef
% vormen. De structuur en titels hangen af van je eigen onderzoek. Je kan bv.
% elke fase in je onderzoek in een apart hoofdstuk bespreken.

%\input{...}
%\input{...}
%...

%%=============================================================================
%% Conclusie
%%=============================================================================

\chapter{Conclusie}%
\label{ch:conclusie}

% TODO: Trek een duidelijke conclusie, in de vorm van een antwoord op de
% onderzoeksvra(a)g(en). Wat was jouw bijdrage aan het onderzoeksdomein en
% hoe biedt dit meerwaarde aan het vakgebied/doelgroep? 
% Reflecteer kritisch over het resultaat. In Engelse teksten wordt deze sectie
% ``Discussion'' genoemd. Had je deze uitkomst verwacht? Zijn er zaken die nog
% niet duidelijk zijn?
% Heeft het onderzoek geleid tot nieuwe vragen die uitnodigen tot verder 
%onderzoek?

\lipsum[76-80]



%---------- Bijlagen -----------------------------------------------------------

\appendix

\chapter{Onderzoeksvoorstel}

Het onderwerp van deze bachelorproef is gebaseerd op een onderzoeksvoorstel dat vooraf werd beoordeeld door de promotor. Dat voorstel is opgenomen in deze bijlage.

%% TODO: 
%\section*{Samenvatting}

% Kopieer en plak hier de samenvatting (abstract) van je onderzoeksvoorstel.

% Verwijzing naar het bestand met de inhoud van het onderzoeksvoorstel
%---------- Inleiding ---------------------------------------------------------

\section{Introductie}%
\label{sec:introductie}

%Deep Learning (DL), Machine Learning (ML) en Artificiele Inteligentie (AI) zijn het voorbije decennia enorm veel gegroeid. Waar jaren geleden nog een supercomputer voor nodig was, kan nu op een bovengemiddelde computer. De kracht van Deep Learning zit in de grote hoeveelheid data die een getraind model accuraat en met hoge snelheid kan verwerken. DL, ML en AI zijn te vinden in een breed spectrum van domeinen, waar ze een grote meerwaarde bieden. Zo wil ILVO onderzoek doen naar het verband tussen hyperspectrale beelden van aardappelen en stootblauw. Het algoritme heeft enkel nood aan de pixels die effectief aardappel bevatten, dus zal de rest van de meetopstelling gemaskeerd moeten worden. Dit maskeren wordt gedaan door een \textit{valse} RGB-foto te genereren. Deze RGB-foto wordt gegenereerd aan de hand van 3 gekozen gofllengtes uit ons gemeten spectrum. Vervolgens wordt deze RGB-foto gesegmenteerd, wat betekent dat enkel de nuttige pixels aangeduid worden. Nu de locaties van de pixels van de aardappel gekend zijn kan alles behalve de aardappel afgeschermd worden. Dit afschermen wordt gedaan door een filter die per hyperspectrale afbeelding 100\% van de pixels van de aardappel doorlaat en 0\% van de niet relevante pixels. Naar deze filter zal vanaf nu gerefereerd worden als masker.
%Dit onderzoek tracht aan te tonen dat het proces van de manuele segmentatie van aardappelen geautomatiseerd kan worden door middel van een gefinetuned Deep Learning model. Als gevolg zal het genereren van de maskers geautomatiseerd kunnen worden.

Het ILVO wil onderzoek doen naar het verband tussen hyperspectrale beelden van aardappelen en stootblauw. Om geautomatiseerde analyse te doen op de beelden moet de meetopstelling uit de beelden gemaskeerd worden om enkel de hyperspectrale data van de aardappelen over te houden. Dit maskeren gebeurt tot nu toe manueel. Uit de hyperspectrale data worden 3 golflengtes geselecteerd om een RGB-foto te genereren. Deze RGB-foto's worden vervolgens gesegmenteerd, wat betekent dat enkel de nuttige pixels (die van de aardappel) aangeduid worden. Nu de locaties van de nuttige pixels gekend zijn kan op het volledige hyperspectrale beeld een filter toegepast worden. Deze filter maskeert de rest van de meetopstelling door 100\% van de pixel-data 'door te laten' van de nuttige pixels en 0\% van de nutteloze pixels. Deze filter fungeert als masker voor de hyperspectrale afbeelding. Per aardappel dient zo'n masker gemaakt te worden.
Dit onderzoek tracht aan te tonen dat het proces van manueel segmenteren en dus ook het maken van de maskers geautomatiseerd kan worden. Alsook welke automatiseringsmethode de beste resultaten geeft.

%---------- Stand van zaken ---------------------------------------------------

\section{State-of-the-art}%
\label{sec:state-of-the-art}

%AI is moeilijk weg te denken uit ons dagelijks leven. De toepassingen hiervan zijn breed, gaande van het tonen van persoonlijke advertenties, tot het maken van kunst en het opsporen van kankercellen. De kracht van deze Deep Learning algoritmes is al extensief bewezen. Ook in landbouw heeft AI zijn plaats. Het grote voordeel van de landbouwsector is dat er een massa aan data te verzamelen valt. Metrics zoals vochtniveau en de opbrengst per vierkante meter kunnen verzameld worden met hyperspectrale beelden van de akkers. Al deze data nuttig verwerken kan een landbouwer helpen bij de keuze van efficiënte rijpaden voor tractoren en het efficiënt besproeien van akkers met pesticiden.

Deep Learning (DL), Machine Learning (ML) en Artificiële Intelligentie (AI) zijn voorbije decennia enorm vooruitgegaan. Taken waar jaren geleden nog supercomputers voor nodig waren, kunnen nu uitgevoerd worden op een gemiddelde computer. Deze groei in rekenkracht zorgt ervoor dat deze drie technologieën kunnen ingezet worden om de massa aan data die vandaag de dag wordt verzameld te verwerken. Zo is een ANN getraind om betere voorspellingen te maken over de energiedensiteit van biomassa \autocite{Veza2022}. Hyperspectrale beeldvorming geeft ons de mogelijkheid om een  continuüm aan golflengtes waar te nemen. Voor exploratief onderzoek is dit ideaal, uit de resultaten kan vaak afgeleid worden welke golflengtes de meeste informatie bevatten. In reële toepassingen kan men dan specifiek waarnemingen doen op relevante golflengtes. Zo kon een onderzoek van het ILVO aantonen dat voor het pathogeen A.solani het best waarneembaar is op golflengtes van 750 nm, 550 nm en 680 nm \autocite{Vijver2020}. Domeinen waar potentieel is voor verzameling van een grote hoeveelheid nuttige data zijn goede kandidaten voor analyse van die data door een Artificial Neural Network (ANN). Hyperspectrale sensoren bieden de mogelijkheid om grote hoeveelheden data te verzamelen. De combinatie van deze twee technologieën lijkt bijgevolg een goede combinatie. In een onderzoek naar het pathogeen P.infestans konden de verschillende stadia van infectie waargenomen en gemonitord worden met een ANN en hyperspectrale beelden \autocite{Wang2008}.

%---------- Methodologie ------------------------------------------------------
\section{Methodologie}%
\label{sec:methodologie}

%Onderstaande stappen worden allemaal uitgevoerd in python. Spectral Python wordt gebruikt voor het inlezen en verwerken van de hyperspectrale data. PyTorch, samen met de Hugging Face libraries wordt aangewend voor het fine-tunen van het Deep Learning model.

%Een Deep Learning algoritme maakt keuzes op basis van invoer of inputs. Deze inputs worden vermenigvuldigd door een bepaald getal, vaak weight genoemd. Vereenvoudigd vermenigvuldigt dit algoritme elk input en weight paar en maakt hier een sommatie van. Deze som word in een non-lineare funtie gestoken waaruit we een relevant eindresultaat krijgen. De weights bepalen de relevantie van de inputs en zijn dus verantwoordelijk voor de accuraatheid van ons model.

%Bijvoorbeeld, als twee weights in functie van het verlies of loss stellen, dit zijn  foute voorspellingen van het algoritme, krijgen we een gemakkelijk te interpreteren 3D-reliëf. In dit virtueel reliëf werkt het algoritme dus het best als we in het diepste dal van loss zitten. Bij meer inputs blijft het principe hetzelfde maar is dit reliëf moeilijker visualiseerbaar omdat we in hogere dimensies werken. Het algoritme werkt naar dit dal toe door de weights volgens de gradiënt van het reliëf aan te passen. Het aanpassen van deze weights volgens de gradiënt is wat er effectief gebeurt als we het model trainen.

%Om het Deep Learning model te trainen moet eerst aan het model duidelijk gemaakt worden wat nu precies een aardappel is en hoe het model zelf aardappelen kan leren herkennen. Dit doen we aan de hand van een handmatig gesegmenteerde dataset van aardappelen

%Dit onderzoek maakt gebruik van een voorgetrained algoritme. Dit is gepubliceerd door Facebook en te vinden op het Hugging Face platform. Het model in kwestie \textit{detr-resnet-50-panoptic} is een panoptisch Segmentatiemodel dat gebruik maakt van End-to-End Object Detection \autocite{Carion2020}. Concreet betekent dit dat het model een combinatie van semantische en instance segmentatie uitvoert. Semantische segmentatie zal elke pixel in het beeld benoemen en een betrouwbaarheidsscore geven aan deze benoeming. Instance segmentatie zal proberen een object te herkennen en enkel de pixels die volgens het algoritme bij dat object horen benoemen. Dit zorgt er ook voor dat elk object ook individueel zal benoemd worden. Bij een reeks aardappelen kan zo onderscheid gemaakt worden tussen elke individuele aardappel of instance van een aardappel.

%De dataset waarop het model zal getrained worden bestaat uit aardappelen die gescheiden in een goot zijn ingescand. Het model zou dus geen probleem mogen hebben met het onderscheiden van de verschillende instances van aardappelen.

%De dataset zal opgesplitst worden in drie subsets: train-, validation- en testdataset. De traindataset zal gebruikt worden in het finetunen van het model. De Validationdata zorgt voor een absolute bron van waarheid tijdens het trainen. Al wordt ook een bias opgebouwd voor de data in de evaluatiedataset, aangezien de weights van het model ook worden aangepast op basis van deze data. Tot slot hebben we de testdataset die we gebruiken om een volledig unbiased evaluatie uit te voeren op de finale waarden van het getrainde model.

Onderstaande stappen worden allemaal uitgevoerd in Python. Python heeft een groot aanbod aan packages, libraries en frameworks die het gemakkelijker maken om een ANN te trainen. De grote community rond ANN's in python is grotendeels te danken aan de toegankelijkheid van de taal en dit zorgt voor heel grondige documentatie en begeleiding.

Gegeven is een dataset van 5 aardappelrassen gescand door twee verschillende hyperspectrale sensoren op twee verschillende dagen. De hyperspectrale sensoren nemen golflengtes van 400nm tot 1700nm waar. Deze data wordt manueel gesegmenteerd en gelabeld.

Dit onderzoek zal twee categorieën van segmentatiemethoden vergelijken met elkaar en de referentie dataset van manueel gesegmenteerde aardappelen. Deze eerste categorie bestaat uit klassieke methodes om aan segmentatie te doen. De tweede categorie vergelijkt verschillende Deep Learning modellen met elkaar.

\textbf{Categorie 1:}

De simpelste techniek voor segmentatie van afbeeldingen is tresholding. Deze techniek tracht objecten van zijn achtergrond te onderscheiden door de assumptie te maken dat vanaf een bepaalde waarde wordt overschreden de pixels deel uit maakt van een bepaald object.

\textbf{Categorie 2:}

Een ANN maakt keuzes op basis van inputs. Deze inputs worden vermenigvuldigd door een bepaalde weight. Vereenvoudigd vermenigvuldigt het algoritme elke input met zijn respectievelijk gewicht en maakt een sommatie van alle input-gewicht paren.  Deze Som wordt in een non-lineaire functie gestoken die een relevant eindresultaat geeft. De weights bepalen de relevantie van de inputs en zijn verantwoordelijk voor de accuraatheid van het model.

Foute voorspellingen van het model zien we als een verlies of loss. De loss van het model is een functie van de weights. Het minima van deze loss-functie is het optimale scenario voor het model. Waar het model het accuraatst is. Dit is een complex gegeven maar in essentie willen we de weights per trainingscyclus (of epoch) van het model zo aanpassen dat we on naar dit minima bewegen. Deze aanpassing van de weights om een minimale foutmarge te verkrijgen is wat concreet gebeurt tijdens het trainen van het model.

De dataset zal opgesplitst worden in drie subsets: train-, validation- en testdataset. De traindataset zal gebruikt worden in het finetunen van de modellen. De validationdata zorgt voor een absolute bron van waarheid tijdens het trainen. Tot slot hebben we de testdataset die we gebruiken om een volledig unbiased evaluatie uit te voeren op de finale waarden van het getrainde model.


Segmentatie heeft verschillende varianten de meest voorkomende zijn semantische- en instance- segmentatie. Semantische segmentatie zal elke pixel in het beeld benoemen en een betrouwbaarheidsscore geven aan deze benoeming. Instance segmentatie zal proberen een object te herkennen en enkel de pixels die volgens het algoritme bij dat object horen benoemen. Dit zorgt er ook voor dat elk object ook individueel zal worden benoemd. Bij een reeks aardappelen kan zo onderscheid gemaakt worden tussen elke individuele aardappel of instance van een aardappel.

Dit onderdeel van het onderzoek maakt gebruik van enkele verschillende modellen:
- detr-resnet-50-panoptic
- official YOLOv7

\textit{detr-resnet-50-panoptic}: een panoptisch Segmentatiemodel dat gebruikmaakt van End-to-End Object Detection \autocite{Carion2020}. Concreet betekent dit dat het model een combinatie van semantische en instance segmentatie kan uitvoeren. 

\textit{Official YOLOv7}: een voorgetraind model origineel gemaakt voor het detecteren van objecten in real time. Dit model kan ook getraind worden op instance segmentatie.


%---------- Verwachte resultaten ----------------------------------------------
\section{Verwacht resultaat, conclusie}%
\label{sec:verwachte_resultaten}

Thresholding zal de snelste manier zijn om aan automatische segmentatie te doen maar zal minder accuraat zijn dan de Deep Learning modellen. De modellen zullen eens getraind snel een accurate segmentatie uitvoeren. De methodes gebruikt in dit onderzoek zouden op een gelijkaardige manier kunnen toegepast worden om andere objecten te segmenteren.



%%---------- Andere bijlagen --------------------------------------------------
% TODO: Voeg hier eventuele andere bijlagen toe. Bv. als je deze BP voor de
% tweede keer indient, een overzicht van de verbeteringen t.o.v. het origineel.
%\input{...}

%%---------- Backmatter, referentielijst ---------------------------------------

\backmatter{}

\setlength\bibitemsep{2pt} %% Add Some space between the bibliograpy entries
\printbibliography[heading=bibintoc]

\end{document}
